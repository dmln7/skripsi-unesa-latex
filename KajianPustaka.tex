\chapter{KAJIAN PUSTAKA}
Bab kajian pustaka bukan sekadar kumpulan kutipan, tetapi kutipan dan teori yang dibahas dan disintesis oleh peneliti/mahasiswa sehingga dapat memunculkan definisi, pemahaman baru, kerangka pikir, hipotesis dan/atau pertanyaan penelitian, serta mengembangkan instrumen yang sesuai dengan permasalahan yang diteliti. Secara umum, bab ini berisi landasan teori, kajian hasil penelitian yang relevan,

\section{Kajian Teori}
Kajian Teori menguraikan teori-teori terkait variabel penelitian meliputi definisi, konsep, asumsi, dan indikator yang digunakan untuk mengukur variabel tersebut dan sebagai landasan untuk mengembangkan instrumen penelitian. Kajian teori diperoleh dari literatur dan hasil penelitian yang relevan. Sumber rujukan untuk kajian teori dapat berupa buku teks, ensiklopedia, kamus, jurnal ilmiah, laporan penelitian, makalah seminar, prosiding, tesis ataupun disertasi. Artikel dalam internet juga dapat digunakan sebagai sumber rujukan apabila artikel ini dimuat dalam pusat-pusat kajian atau ditulis oleh penulis bereputasi. Namun, materi pembelajaran tidak dapat digunakan sebagai sumber rujukan karena belum mengalami uji publik melalui publikasi.

\subsection{Notasi Komutasi}
Jika simbol \( (m) \) ditambahkan ke sudut kanan atas, itu menunjukkan \textbf{nilai sekarang (Present Value, PV) dari suatu anuitas} di mana pembayaran dilakukan setiap \( \frac{1}{m} \) dari satu tahun selama \( n \) tahun, dan setiap pembayaran besarnya \( \frac{1}{m} \) dari satu unit.


\begin{align*}
	\ax{\angln i}[\left(m\right)] &= \frac{1 - v^n}{i^{(m)}}\\
	\ax**{\angln i}[\left(m\right)]  &= \frac{1 - v^n}{d^{(m)}}
\end{align*}


Di mana:
\begin{itemize}
	\item $\ax{\angln i}[\left(m\right)]$ menyatakan \textbf{nilai sekarang anuitas biasa} (anuitas-immediate), yaitu anuitas yang dibayar setiap \( \frac{1}{m} \) dari satu tahun pada akhir setiap periode.
	\item $\ax**{\angln i}[\left(m\right)]$ menyatakan \textbf{nilai sekarang anuitas jatuh tempo} (anuitas-due), yaitu anuitas yang dibayar di awal setiap periode.
	\item $i^{(m)}$ adalah \textbf{tingkat bunga nominal tahunan yang dikapitalisasi $m$ kali dalam setahun}.
	\item $d^{(m)}$ adalah \textbf{tingkat diskonto nominal tahunan yang dikapitalisasi $m$ kali dalam setahun}.
	\item $v = \frac{1}{1 + i}$ adalah faktor diskonto (discount factor), yang digunakan untuk menghitung nilai sekarang dari suatu pembayaran yang akan diterima di masa depan.
\end{itemize}

Dengan menggunakan rumus ini, kita dapat menghitung \textbf{nilai sekarang dari anuitas dengan pembayaran lebih sering} dari satu kali dalam setahun, seperti anuitas bulanan, kuartalan, atau semi-tahunan.


\begin{figure}[h!]
	\centering
	\includegraphics[width=0.5\textwidth]{Gambar/logounesa}
	\caption{Logo Unesa}
	\label{fig:logounesa}
\end{figure}

\subsubsection{Rumus}
Rumus umum persamaan pythagoras diberikan oleh persamaan~\ref{eq:pythagoras} berikut ini
\begin{equation}
	a^2 + b^2 = c^2 \label{eq:pythagoras}
\end{equation}

Model penyebaran penyakit diberikan oleh sistem persamaan diferensial sebagai berikut:
\begin{align}
	\frac{dS}{dt} &= \beta S I\\
	\nonumber \frac{dI}{dt} &= -\beta S I
\end{align}

\begin{equation}
	\begin{aligned}
		\frac{dS}{dt} &= \beta S I\\
		\frac{dI}{dt} &= -\beta S I
	\end{aligned}
\end{equation}

\begin{equation}
	I = \int_0^{\infty} e^{at}~{dt}
\end{equation}

Matriks Identitas $3 \times 3$ diberikan oleh:
\begin{equation}
	I = \begin{bmatrix}
		1 & 0 & \cdots & 0\\
		0 & 1 & \cdots & \vdots\\
		\cdots & \cdots & \ddots &\vdots\\
		0 & 0 & \cdots & 1
	\end{bmatrix}
\end{equation}

\begin{theorem}[Teorema Keterbagian]
	Diberikan bilangan bulat  $a$, $b$, dan $c$ dengan $a \neq 0$ sehingga berlaku sifat-sifat berikut ini.
\end{theorem}

\begin{theorem}[Teorema Keterbagian]
	Diberikan bilangan bulat  $a$, $b$, dan $c$ dengan $a \neq 0$ sehingga berlaku sifat-sifat berikut ini.
\end{theorem}

\begin{proof}
	Berdasarkan $\cdots$
\end{proof}

\begin{definition}
	Diberikan bilangan bulat $a$ dan $b$ dengan $a \neq 0$. Jika $b$ merupakan kelipatan dari $a$, maka kita katakan bahwa  $a$ habis membagi $b$ atau dinotasikan $a|b$
\end{definition}

\begin{example}
	Diberikan bilangan bulat $a$ dan $b$ dengan $a \neq 0$. Jika $b$ merupakan kelipatan dari $a$, maka kita katakan bahwa  $a$ habis membagi $b$ atau dinotasikan $a|b$
\end{example}

\begin{theorem}
	Diberikan bilangan bulat  $a$, $b$, dan $c$ dengan $a \neq 0$ sehingga berlaku sifat-sifat berikut ini.
\end{theorem}

\section{Hasil Penelitian yang Relevan}
Hasil Penelitian yang relevan berfungsi memperkuat posisi penelitian yang dilakukan saat ini dengan melihat hasil-hasil penelitian yang sudah dilakukan. Hasil penelitian yang relevan juga digunakan sebagai dasar peneliti menyusun kerangka berpikir. Hasil penelitian yang relevan disajikan secara narasi dengan menganalisis hasil penelitian yang satu dengan hasil penelitian yang lain.

\section{Kerangka Berpikir}
Kerangka Berpikir berisikan gambaran logis dan rasional tentang bagaimana variable-variabel penelitian dapat saling berhubungan (korelasi). Kerangka berpikir akan mengarahkan peneliti kepada perumusan hipotesis. Penelitianyang tidak membuktikan hipotesis seperti penelitian dengan pendekatan kualitatif, tidak perlu menuliskan kerangka berpikir.

\section{Pertanyaan Penelitian dan/atau Hipotesis (jika ada)}
Pertanyaan penelitian merupakan penegasan dari rumusan masalah yang akan dicari jawabannya melalui penelitian. Hipotesis merupakan jawaban sementara terhadap rumusan masalah yang dinyatakan dengan kalimat pernyataan. Untuk penelitian yang tidak membuktikan hipotesis cukup menuliskan pertanyaan penelitian. Hipotesis atau pertanyaan penelitian harus selaras dan merupakan jabaran dari rumusan masalah.