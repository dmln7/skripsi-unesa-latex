\chapter{PENDAHULUAN}

\section{Latar Belakang}
Latar Belakang Masalah menjelaskan alasan-alasan rasional yang melandasi pentingnya penelitian tersebut dilakukan. Untuk membuat alasan rasional perlu diungkapkan kesenjangan antara kenyataan yang terjadi dibandingkan kenyataan yang diharapkan. Berbagai data, fakta, pendapat, keluhan dari lapangan/tempat penelitian perlu diungkap untuk memperkuat alasan perlunya dilakukan penelitian~\cite{Ver}

\section{Identifikasi Masalah}
Identifikasi Masalah menjelaskan kajian berbagai kemungkinan penyebab terjadinya masalah. Pada bagian ini perlu diungkap secara luas berbagai permasalahan yang mungkin untuk diteliti. Isi identifikasi masalah harus selaras dengan masalah yang diungkapkan pada latar belakang masalah.

\section{Batasan Masalah}
Batasan Masalah yakni penetapan masalah (dari berbagai masalah yang teridentifikasi) dengan mempertimbangkan berbagai aspek metodologis, kelayakan untuk diteliti, serta keterbatasan peneliti tanpa mengorbankan kebermaknaan arti, konsep, atau topik yang diteliti. Misalnya
\begin{daftar}
	\item Penelitian dilakukan dalam dalam bentuk simulasi numerik di komputer.
	\item Data kecepatan yang dipergunakan dalam penelitian ini diambil dari PT. Jasa Tirta I.
	\item Unsur-unsur hidrodinamika yang diteliti adalah kecepatan aliran.
	\item Pola penyebaran polutan yang diamati adalah arah panjang sungai (longitudinal) dan arah lebar sungai (lateral) selama tahun 2012.
	\item Parameter kualitas sungai yang digunakan adalah TSS (\textit{Total Suspended Solid}).
	\item Aliran sungai ditentukan bersifat kondisi Laminer.
	\item Kepadatan air sungai konstan karena air sungai adalah fluida yang tidak mampu mampat.
	\item Perubahan viskositas air cukup kecil sehingga dianggap konstan.
	\item Permukaan sungai adalah horizontal dan dinding sungai berkarakteristik relatif halus.
	\item Air sungai mengandung polutan TSS, dan polutan TSS menyebar mengikuti kecepatan aliran sungai.
	\item Pengaruh putaran bumi (gaya \textit{Coriolis}) sangat kecil sehingga dianggap nol.
	\item Gradien tekanan pada masing-masing sumbu ditentukan.
	\item Pengaruh angin sangat kecil sehingga gesekan di permukaan diasumsikan nol.
	\item Panjang sungai yang diukur dari pertemuan dua sungai adalah $1500m$ dan lebarnya $25m$
	\item Sungai yang menjadi objek penelitian ini adalah Kali Surabaya yang mengalir diantara Jalan Raya Mastrip (Karangpilang, Surabaya) dan Jalan Ngelom Rolak (Sepanjang, Sidoarjo)
\end{daftar}

\section{Rumusan Masalah}
Rumusan masalah berisi penegasan masalah yang akan diteliti sebagai hasil dari pembatasan masalah-masalah yang teridentifikasi. Rumusan masalah dituliskan dalam kalimat tanya. Misalnya:
\begin{daftar}
	\item Bagaimana model matematika penyebaran polutan pada pertemuan dua sungai.
	\item Bagaimana menerapkan Metode Volume Hingga skema \textit{QUICK} pada model penyebaran polutan pada pertemuan dua sungai tersebut.
	\item Bagaimana hasil penyebaran polutan di daerah aliran pertemuan dua sungai dengan Metode Volume Hingga skema \textit{QUICK}.
\end{daftar}

\section{Tujuan Penelitian}
Tujuan Penelitian menyatakan target yang akan dicapai melalui penelitian. Tujuan dirumuskan selaras/mengacu kepada rumusan masalah. Misalnya
\begin{daftar}
	\item Mengkaji dan menganalisis model matematika penyebaran polutan pada pertemuan dua sungai.	
	\item Menerapkan Metode Volume Hingga skema \textit{QUICK} dan menyelesaikan model matematika penyebaran polutan pada pertemuan dua sungai tersebut.
	\item Menyimulasikan dan memvisualisasikan penyelesaian numerik pola penyebaran polutan pada pertemuan dua sungai.
\end{daftar}

\section{Manfaat Penelitian}
Manfaat Penelitian menjelaskan manfaat hasil penelitian untuk kepentingan teoretis maupun praktis. Misalnya
\begin{daftar}
	\item Manfaat Teoretis 
	\item Manfaat Praktis
	\begin{daftar}
		\item Bagi Penulis
		\item Bagi Peneliti selanjutnya
		\item Bagi \textit{stakeholder} terkait (sebutkan \textit{stakeholder}nya)
	\end{daftar}
\end{daftar}

\section{Asumsi penelitian (jika ada)}
Asumsi penelitian (jika ada) adalah anggapan dasar tentang suatu hal yang dijadikan pijakan berpikir dan bertindak dalam melaksanakan penelitian. Asumsi dapat juga diartikan sebagai anggapan dasar yang menyebabkan suatu teori dapat berlaku. Asumsi dapat bersifat substantif atau metodologis. Asumsi substantif berkenaan dengan permasalahan penelitian, sedangkan asumsi metodologis berkenaan dengan metode penelitian.

